\documentclass{article}
\usepackage{bytefield}
\usepackage{hyperref}

\title{CS244B Mazewar Protocol Specification}
\author{Ned Bass      \texttt{<nedbass@stanford.edu>} \\
        Prakash Surya \texttt{<me@prakashsurya.com>}}
\date{\today}

\begin{document}
\maketitle

\section{Introduction}
CS244B Mazewar is a distributed, multiplayer game that allows each
player to control a rat in a maze. Each player receives points for
tagging other players with a projectile, and loses points for being
tagged and shooting projectiles. Due to it's distributed nature, the
game is fault-tolerant, as players can continuously leave and join the
game without disrupting other players.

\section{Protocol Description}
The Mazewar protocol defines the way in which each instance of
the game communicate over the network, and is intended to be implemented
over an unreliable transport layer such as UDP. In addition, the
protocol assumes all players (including new players trying to join) see
all Mazewar communication sent over the network via multicast. The
protocol consists of two distinct phases of communication, and a set
of well defined packet types.

\subsection{The Discovery Phase}
The discovery phase provides new players the ability to discover any
currently active game. While in this phase,
the client does not send any outbound traffic and listens for
a minimum of 5 seconds for any incoming Mazewar
traffic from other active players. If there is no incoming traffic
during this time, it is assumed there isn't a current game being played
on the network.

\subsection{The Active Phase}
During the active phase a player is actively participating
in a Mazewar game. While in this phase, each client is
capable of sending 5 types of messages; a ``State" message, a
``Nickname" message, a ``Tagged" message, the ``Tagged Acknowledgment"
message, and an ``Exit" message.

\section{Packet Definitions}
The following packet types are defined by the protocol: \\ \\
\begin{tabular}{|l|p{.7\textwidth}|}
	\hline
	Descriptor & Description \\ \hline \hline
	State      & Communicates position and direction of the rat,
	             position and direction of the projectile, and
	             the player's score. \\ \hline
	Nickname   & Communicates the nickname and GUID. \\ \hline
	Tagged     & Sent when the local rat has been hit by a remote
	             rat's projectile. This message must be acknowledged. \\ \hline
	Tagged ACK & Acknowledge receipt of a tagged packet. \\ \hline
	Leaving    & Advertises a player leaving the game. \\ \hline
	Request for Retransmission & Requests a client to retransmit a
	                             packet \\ \hline
\end{tabular}

\subsection{Packet Header}
\begin{figure}[htbp]
\centering
	\begin{bytefield}{32}
		\bitheader{0,7-8,15-16,23-24,31} \\
		\bitbox{8}{Descriptor} & \bitbox{24}{Must Be Zero} \\
		\wordbox{2}{Globally Unique Identifier (GUID)} \\
		\wordbox{2}{Sequence Number}
	\end{bytefield}
	\caption{Packet Header}
\end{figure}

\subsubsection{Descriptor}

\begin{tabular}{|l|l|}
	\hline
	Value & Type \\ \hline \hline
	0 & State packet \\ \hline
	1 & Nickname packet \\ \hline
	2 & Tagged packet \\ \hline
	3 & Tagged ACK \\ \hline
	4 & Leaving packet \\ \hline
	5 & Reqest for retransmission packet \\ \hline
\end{tabular}

\subsubsection{GUID}
A randomly generated 64 bit identifier used to distinguish clients. This
is determined upon joining a game and must not change during a single
session. The probability of a GUID collision is assumed to be
negligible.

\subsubsection{Sequence Number}
A monotonically increasing number that uniquely identifies a packet
transmitted by a client. This must be incremented by one for each packet
transmitted unless the packet is a retransmission. This can be used for
detecting out of order packet delivery, dropped packets, and for
requesting retransmission.

\subsection{State Packet}
\begin{figure}[htbp]
\centering
	\begin{bytefield}{32}
		\bitheader{0,7-8,15-16,23-24,31} \\
		\wordbox{5}{Packet Header} \\
		\bitbox{15}{Rat Position X} & \bitbox{15}{Rat Position Y} &
			\bitbox{2}{RD} \\
		\bitbox{15}{Projectile Position X} &
			\bitbox{15}{Projectile Position Y} &
			\bitbox{2}{PD} \\
		\wordbox{1}{Player Score} \\
		\wordbox{2}{Timestamp} \\
		\wordbox{2}{Collision Resolution Token (CRT)}
	\end{bytefield}
	\caption{State Packet}
\end{figure}

\subsubsection{Rat Position X, Rat Position Y, and Rat Direction (RD)}
The maze coordinates of the rat and the direction it is facing.

\subsubsection{Projectile Position X, Projectile Position Y, and
               Projectile Direction (PD)}
The maze coordinates of the rat's projectile and the direction it is
travelling. If no projectile is active, the word containing these
fields must be \texttt{0xffffffff}.

\subsubsection{Player Score}
A signed value representing the player's current score.

\subsubsection{Timestamp}
Represents the global game time the packet was transmitted. See Section
\ref{ssec:timestamp} for details on game time management.

\subsubsection{Collision Resolution Token}
A 64 bit value randomly generated for each packet. This is used to
resolve timing conflicts. See Sections \ref{ssec:collision} and
\ref{ssec:tagging} for further details.

\subsection{Nickname Packet}
\begin{figure}[htbp]
\centering
	\begin{bytefield}{32}
		\bitheader{0,7-8,15-16,23-24,31} \\
		\wordbox{5}{Packet Header} \\
		\wordbox{8}{Nickname}
	\end{bytefield}
	\caption{Nickname Packet}
\end{figure}

\subsubsection{Nickname}
A 32 character null terminated string representing the player's
nickname.

\newpage
\subsection{Tagged Packet}
\begin{figure}[htbp]
\centering
	\begin{bytefield}{32}
		\bitheader{0,7-8,15-16,23-24,31} \\
		\wordbox{5}{Packet Header} \\
		\wordbox{2}{GUID of Shooter}
	\end{bytefield}
	\caption{Tagged Packet}
\end{figure}

\subsubsection{GUID of Shooter}
The GUID of the client that tagged the local rat.

\subsection{Tagged Acknowledgment Packet}
\begin{figure}[htbp]
\centering
	\begin{bytefield}{32}
		\bitheader{0,7-8,15-16,23-24,31} \\
		\wordbox{5}{Packet Header} \\
		\wordbox{2}{GUID of Tagged Rat} \\
		\wordbox{2}{Tagged Packet Sequence Number}
	\end{bytefield}
	\caption{Tagged ACK Packet}
\end{figure}

\subsubsection{GUID of Tagged Rat}
The GUID of the client being acknowledged.

\subsubsection{Tagged Packet Sequence Number}
The sequence number of the packet being acknowledged.

\subsection{Leaving Packet}
\begin{figure}[htbp]
\centering
	\begin{bytefield}{32}
		\bitheader{0,7-8,15-16,23-24,31} \\
		\wordbox{5}{Packet Header} \\
		\wordbox{2}{GUID of Leaving Client}
	\end{bytefield}
	\caption{Leaving Packet}
\end{figure}

\subsubsection{GUID of Leaving Client}
The GUID of the client that is leaving the game.

\subsection{Request for Retransmission Packet}
\begin{figure}[htbp]
\centering
	\begin{bytefield}{32}
		\bitheader{0,7-8,15-16,23-24,31} \\
		\wordbox{5}{Packet Header} \\
		\wordbox{2}{GUID of Requestee} \\
		\wordbox{2}{Requested Packet Sequence Number} \\
	\end{bytefield}
	\caption{Request for Retransmission Packet}
\end{figure}

\subsubsection{GUID of Requestee}
The GUID of the client being asked to retransmit.

\subsubsection{Requested Packet Sequence Number}
The sequence number of the packet being requested.

\section{Timing and Semantics}

\subsection{Player Moves}
The client must send out a \textit{state packet} to all other players for each
event causing it to change its local state. Local state includes
player position and direction, projectile position and direction, and
score. In addition, it must send at least one state packet every 500ms.

\subsection{Collisions}
\label{ssec:collision}
A collision is defined as two or more objects attempting to occupy the
same cell at the same time. An object may be a rat or a projectile. Two
events are defined to occur at the \textit{same time} if their
timestamps are within 250ms of each other.

\subsection{Player Movement Collision Resolution}
If a collision occurs between two rats, the contended tile will be
occupied by the rat whose \textit{state packet} contained the lowest CRT. The
other rat must revert to its previous position.

\subsubsection{Tag Detection}
\label{ssec:tagging}
A tag occurs when a projectile collides with a rat. The determination
that a tag has occurred is made at the tagged client, and this client is
responsible for notifying the shooter via a \textit{tagged packet}.
The tagged client must continue to retransmit the \textit{tagged packet}
every 500ms until an acknowledgement is received or the shooter has left
the game.

\subsubsection{Contended Tag Resolution}
If there is ambiguity regarding which remote client tagged the local
client then method of awarding the tag is implementation defined, so
long as exactly one shooter is selected.

\subsection{Joining a Game}
\subsection{Leaving a Game}
\subsection{Possible Inconsistencies}
\subsection{Loss of Contact}
\subsection{Game Time Management}
\label{ssec:timestamp}

\end{document}
