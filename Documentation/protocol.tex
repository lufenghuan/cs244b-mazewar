\documentclass{article}
\usepackage{bytefield}
\usepackage{hyperref}

\title{CS244B Mazewar Protocol Specification}
\author{Ned Bass      \texttt{<nedbass@stanford.edu>} \\
        Prakash Surya \texttt{<surya1@stanford.edu>}}
\date{\today}

\begin{document}
\maketitle

\section{Introduction}
CS244B Mazewar is a distributed, multiplayer game that allows each
player to control a rat in a maze. Each player receives points for
tagging other players with a projectile, and loses points for being
tagged and shooting projectiles. Due to it's distributed nature, the
game is fault-tolerant, as players can continuously leave and join the
game without disrupting other players.

\section{Protocol Description}
The Mazewar protocol defines the way in which each instance of the game
communicates over the network, and is intended to be implemented over an
unreliable transport layer such as UDP. In addition, the protocol
assumes all players (including new players trying to join) see all
Mazewar communication sent over the network via multicast. The protocol
consists of two distinct phases of communication, and a set of well
defined packet types.

\subsection{The Discovery Phase}
\label{ssec:discovery}
The discovery phase provides new players the ability to discover any
currently active game.  This phase is also used to learn the global game
time, which is discussed in Section \ref{ssec:gametime}.  While in this
phase, the client does not send any outbound traffic and listens for a
minimum of 5 seconds for any incoming Mazewar traffic from other active
players. If there is no incoming traffic during this time, it is assumed
there isn't a current game being played on the network.

\subsection{The Active Phase}
During the active phase a player is actively participating in a Mazewar
game. While in this phase, each client may send any of the packet types
defined in Section \ref{sec:packetdefs}.

\section{Packet Definitions}
\label{sec:packetdefs}
The following packet types are defined by the protocol: \\ \\
\begin{tabular}{|l|p{.7\textwidth}|}
	\hline
	Descriptor & Description \\ \hline \hline
	State      & Communicates position and direction of the rat,
	             position and direction of the projectile, and
	             the player's score. \\ \hline
	Nickname   & Communicates the nickname and GUID. \\ \hline
	Tagged     & Sent when the local rat has been hit by a remote
	             rat's projectile. This message must be acknowledged. \\ \hline
	Tagged ACK & Acknowledge receipt of a tagged packet. \\ \hline
	Leaving    & Advertises a player leaving the game. \\ \hline
	Request for Retransmission & Requests a client to retransmit a
	                             packet \\ \hline
\end{tabular}

\subsection{Packet Header}
\begin{figure}[htbp]
\centering
	\begin{bytefield}{32}
		\bitheader{0,7-8,15-16,23-24,31} \\
		\bitbox{8}{Descriptor} & \bitbox{24}{Must Be Zero} \\
		\wordbox{2}{Globally Unique Identifier (GUID)} \\
		\wordbox{2}{Sequence Number}
	\end{bytefield}
	\caption{Packet Header}
\end{figure}

\subsubsection{Descriptor}

\begin{tabular}{|l|l|}
	\hline
	Value & Type \\ \hline \hline
	0 & State packet \\ \hline
	1 & Nickname packet \\ \hline
	2 & Tagged packet \\ \hline
	3 & Tagged ACK \\ \hline
	4 & Leaving packet \\ \hline
	5 & Reqest for retransmission packet \\ \hline
\end{tabular}

\subsubsection{Must Be Zero}
A 24-bit field reserved for future use, such as for a protocol version
number.

\subsubsection{GUID}
A randomly generated 64-bit identifier used to distinguish clients. This
is determined upon joining a game and must not change during a single
session. The probability of a GUID collision is assumed to be
negligible so no provision is made for resolving conflicts.

\subsubsection{Sequence Number}
A monotonically increasing number that uniquely identifies a packet
transmitted by a client. This must be incremented by one for each packet
transmitted unless the packet is a retransmission. This can be used for
detecting out of order packet delivery, dropped packets, and for
requesting retransmission.

\subsection{State Packet}
\begin{figure}[htbp]
\centering
	\begin{bytefield}{32}
		\bitheader{0,7-8,15-16,23-24,31} \\
		\wordbox{5}{Packet Header} \\
		\bitbox{15}{Rat Position X} & \bitbox{15}{Rat Position Y} &
			\bitbox{2}{RD} \\
		\bitbox{15}{Projectile Position X} &
			\bitbox{15}{Projectile Position Y} &
			\bitbox{2}{PD} \\
		\wordbox{1}{Player Score} \\
		\wordbox{2}{Timestamp} \\
		\wordbox{2}{Collision Resolution Token (CRT)}
	\end{bytefield}
	\caption{State Packet}
\end{figure}

\subsubsection{Rat Position X, Rat Position Y, and Rat Direction (RD)}
The maze coordinates of the rat and the direction it is facing.

\subsubsection{Projectile Position X, Projectile Position Y, and
               Projectile Direction (PD)}
The maze coordinates of the rat's projectile and the direction it is
travelling. If no projectile is active, the word containing these
fields must be \texttt{0xffffffff}.

\subsubsection{Player Score}
A signed value representing the player's current score.

\subsubsection{Timestamp}
Represents the global game time $G$ at which the packet was transmitted.
$G$ is defined as
\[G=E+t\] where $E$ is the local epoch (see Section
\ref{ssec:gametime}), and $t$ is the number of milliseconds elapsed
since $E$ was initialized.

\subsubsection{Collision Resolution Token}
A 64-bit value randomly generated for each packet. This is used to
resolve timing conflicts. See Sections \ref{ssec:collision} and
\ref{ssec:tagging} for further details.

\subsection{Nickname Packet}
\begin{figure}[htbp]
\centering
	\begin{bytefield}{32}
		\bitheader{0,7-8,15-16,23-24,31} \\
		\wordbox{5}{Packet Header} \\
		\wordbox{8}{Nickname}
	\end{bytefield}
	\caption{Nickname Packet}
\end{figure}

\subsubsection{Nickname}
A 32 character null-terminated string representing the player's
nickname.

\newpage
\subsection{Tagged Packet}
\begin{figure}[htbp]
\centering
	\begin{bytefield}{32}
		\bitheader{0,7-8,15-16,23-24,31} \\
		\wordbox{5}{Packet Header} \\
		\wordbox{2}{GUID of Shooter}
	\end{bytefield}
	\caption{Tagged Packet}
\end{figure}

\subsubsection{GUID of Shooter}
The GUID of the client that tagged the local rat.

\subsection{Tagged Acknowledgment Packet}
\begin{figure}[htbp]
\centering
	\begin{bytefield}{32}
		\bitheader{0,7-8,15-16,23-24,31} \\
		\wordbox{5}{Packet Header} \\
		\wordbox{2}{GUID of Tagged Rat} \\
		\wordbox{2}{Tagged Packet Sequence Number}
	\end{bytefield}
	\caption{Tagged ACK Packet}
\end{figure}

\subsubsection{GUID of Tagged Rat}
The GUID of the client being acknowledged.

\subsubsection{Tagged Packet Sequence Number}
The sequence number of the packet being acknowledged.

\subsection{Leaving Packet}
\begin{figure}[htbp]
\centering
	\begin{bytefield}{32}
		\bitheader{0,7-8,15-16,23-24,31} \\
		\wordbox{5}{Packet Header} \\
		\wordbox{2}{GUID of Leaving Client}
	\end{bytefield}
	\caption{Leaving Packet}
\end{figure}

\subsubsection{GUID of Leaving Client}
The GUID of the client that is leaving the game.

\subsection{Request for Retransmission Packet}
\begin{figure}[htbp]
\centering
	\begin{bytefield}{32}
		\bitheader{0,7-8,15-16,23-24,31} \\
		\wordbox{5}{Packet Header} \\
		\wordbox{2}{GUID of Requestee} \\
		\wordbox{2}{Requested Packet Sequence Number} \\
	\end{bytefield}
	\caption{Request for Retransmission Packet}
\end{figure}

\subsubsection{GUID of Requestee}
The GUID of the client being asked to retransmit.

\subsubsection{Requested Packet Sequence Number}
The sequence number of the packet being requested.

\section{Timing and Semantics}

\subsection{Player Moves}
The client must send out a \textit{state packet} to all other players for each
event causing it to change its local state. Local state includes
player position and direction, projectile position and direction, and
score. In addition, it must send at least one state packet every 500ms.

\subsection{Collisions}
\label{ssec:collision}
A collision is defined as two or more objects attempting to occupy the
same cell at the same time. An object may be a rat or a projectile. Two
events are defined to occur at the \textit{same time} if their
timestamps are within 250ms of each other.

\subsection{Player Movement Collision Resolution}
If a collision occurs between two rats, the contended tile will be
occupied by the rat whose \textit{state packet} contained the lowest CRT. The
other rat must revert to its previous position.

\subsubsection{Tag Detection}
\label{ssec:tagging}
A tag occurs when a projectile collides with a rat. The determination
that a tag has occurred is made at the tagged client, and this client is
responsible for notifying the shooter via a \textit{tagged packet}.
The tagged client must continue to retransmit the \textit{tagged packet}
every 500ms until an acknowledgement is received or the shooter has left
the game.

\subsubsection{Contended Tag Resolution}
If there is ambiguity regarding which remote client tagged the local
client then the method of awarding the tag is implementation defined, so
long as exactly one shooter is selected.

\subsection{Joining a Game}
A client joining a game discovers the state of existing players during
the \textit{discovery phase} as described in Section \ref{ssec:discovery}.
Upon transition to the \textit{active phase} the newly joined client
begins transmitting packets.  Other clients dynamically discover the new
client by receiving its traffic.

\subsection{Leaving a Game}
A client exiting a game must transmit a \textit{leaving packet} which
need not be acknowledged.  If no traffic is received from a client for
10 seconds then that client is assumed to be treated by remaining
clients as if it had exited.

\subsection{Nicknames}
The \textit{nickname packet} is used to maintain a mapping of client
GUIDs to player nicknames for user-interface purposes.  Each client must
transmit a \textit{nickname packet} at least once every 5 seconds.  An
implementation is free to define how a player's nickname is displayed if
no mapping yet exists.

\subsection{Temporary Loss of Contact}
Transient network disruptions may cause clients to time out and be
removed from one another's game state.   Because remote client state
is dynamically discovered from received traffic, the game will converge
back to a consistent state when communication is restored.

\subsection{Game Time Management}
\label{ssec:gametime}
A client joining the game establishes a local time epoch during the
discovery phase.  This epoch must be derived from the timestamps of
received state packets.  In this way the local time epoch will be
synchronized with the peer game times within the network latency.  If no
traffic is received during the discovery phase then the client must set
its epoch to zero.  It is assumed that system clock drift between
clients is negligible.

\subsection{Possible Inconsistencies}

\subsubsection{Packet Loss}
Packet loss due to a transient network disruption may lead to temporary
inconsistencies such as undetected collisions.  Such temporary artifacts
are acceptable because the game state will converge back to a consistent
state when the disruption subsides.  This assurance is provided by the
idempotency of state packets since they use absolute position values.

\subsubsection{Out-of-Order Packet Delivery}
Processing state packets out of order could lead to erratic movement of
remote objects.  Therefore packets received with lower sequence numbers
than the most recently processed packet for a given client must be
dropped.

\subsubsection{Clock Drift}
The assumption of negligible clock drift may prove to be unwarranted.
Significant drift could lead undetected collisions.  The simplest way to
handle this problem may be to abort the game if significant time skew is
detected.  Less disruptive but more complex solutions could correct for
drift or identify individual offending clients to evict from the game.

\subsubsection{GUID collisions}
It is technically possible for two clients to randomly generate the same
64-bit GUID.  While exceedingly unlikely if a high quality random number
generator is used, such an event would cause unacceptable global
inconsistencies in the game state.  However, for the sake of simplicity
no provision is made for detecting and resolving GUID conflicts.

\end{document}
