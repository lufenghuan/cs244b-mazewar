\documentclass{article}

\title{CS244B Mazewar Protocol Questions and Answers}
\author{Prakash Surya \texttt{<surya1@stanford.edu>}}
\date{\today}

\begin{document}
\maketitle

\begin{enumerate}

\item Evaluate the portion of your design that deals with starting,
maintaining, and exiting a game - what are its strengths and weaknesses?

The design that my group and I came up with has a number of strengths
including resiliency despite packet loss and reordering. Since all of
the information a client needs from a remote client is sent with each
state packet, consistency remains even with a high amount of packet
loss.

The acknowledgement of "tagged" packets also keep scores consistent even
if a number of these packets are lost. The client will keep
retransmitting, trying to let the other client know he should be awarded
the tag.

If a leaving packet is lost, the timeout mechanism will correct this.

Nickname packets are continuously sent out at regular intervals. This is
unnecessary once a client receives a single nickname packet from a given
client, but it keeps it simple as these packets need not be
acknowledged.

One issue with the design is the amount of communication sent over the
network, even if there is no changes in state. If this design was to
scale to millions of players, the number of packets sent over the
network would need to be reduced to avoid congesting the network (which
could potentially cause increased packet loss, latency, and reordering).

Another big drawback of the protocol is the fact that it doesn't deal
with GUID collisions. Thus if two clients had the same GUID, these
packets would get interpreted as a single player. This is very unlikely
given a decent 64-bit PRNG, but theoretically it is still possible.

\item Evaluate your design with respect to its performance on its current
platform (i.e. a small LAN linked by ethernet). How does it scale for an
increased number of players? What if it is played across a WAN? Or if
played on a network with different capacities?

\item Evaluate your design for consistency. What local or global
inconsistencies can occur? How are they dealt with?

\item Evaluate your design for security. What happens if there are
malicious users?

A malicious user could wreak havoc on the game as there are not any
anti-cheating mechanisms built into the protocol. For example, a user
could set hit score to any arbitrary value, and the remote clients would
accept and trust this value blindly.

One could also send out malicious "tagged" packets, causing non
malicious clients to award themselves tags even when this never
happened. Thus, not only can a malicious user control it's score
arbitrarily, it can manipulate the other clients scores.

Malicious CRT values can be used to win all state change collisions, or
always lose these collisions. This could potentially be used to
manipulate the movements of the other players.

Another big vulnerability in the protocol is GUID spoofing. A malicious
user could send packets with a spoofed GUID value. This could cause
clients to think a new player has joined when this isn't the case, or
worse, cause clients to think an existing player has sent packets which
it hasn't.

\end{enumerate}

\end{document}
